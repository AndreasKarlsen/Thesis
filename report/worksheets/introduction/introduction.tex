\makeatletter \@ifundefined{rootpath}{\input{../../setup/preamble.tex}}\makeatother
\worksheetstart{Introduction}{0}{Februar 10, 2015}{}{../../}
%
\section{Motivation}
Today the increase in \ac{CPU} speed comes in the form of additional cores, and not faster clock speed\cite{sutter2005free}. In order to utilize this increase in speed, many of the popular sequential programming languages, such as C, C++, Java, and C\#, require changes or new additions in order to adapt\cite[p. 56]{sutter2005software}. Our recent study\cite{dpt907e14trending}, analyzed the runtime performance and characteristics of three different approaches to concurrent programming: \ac{TL}, \ac{STM}, and the Actor model. The study concluded that \ac{STM} eliminates many of the risks related to programming in the context of shared-memory concurrency, notably deadlocks, leading to simpler reasoning and improved usability. Additionally, \ac{STM} fits with the thread model used by sequential languages, and can thereby be applied to existing implementations, without requiring major rewrites. The runtime performance of \ac{STM} is at a competitive level compared to fine-grained locking\cite{dpt907e14trending}. The analysis uncovered one major caveat of \ac{STM}, it is none orthogonal with side-effects, such as \ac{IO} and exceptions\cite{dpt907e14trending}.
%Utilizing tomorrows increasingly faster \acp{CPU} requires changes to the sequential programming paradigm\cite[p. 56]{sutter2005software} which is popularized in languages such as C, C++, Java, and C\#.

Despite of the advantages \ac{STM} could provide to sequential languages, \ac{STM} has, as of the time of writing, seen only limited official language integration. Only in Clojure\footnote{\url{http://clojuredocs.org/}} and Haskell\footnote{\url{https://wiki.haskell.org/Haskell}} has \ac{STM} been introduced as a official built-in language feature. \ac{STM} has been introduced as a library based solution in sequential languages, however in order to enable first class support, such as static code analysis and syntax support, language integration is required. With the April 2014 event of open sourcing of the new C\# compiler\cite{roslyn} with codename Roslyn as well as the February 2015 open sourcing of the \ac{CLR}\cite{coreclr}, Microsoft facilitates an opportunity to extend the C\# language and compiler as well as its runtime system. Thereby opening up for integrating \ac{STM} into the language.
%\footnote{\url{https://github.com/dotnet/roslyn}}
%\footnote{\url{https://github.com/dotnet/coreclr}}
\andreas[inline]{Programmer might not want to switch to an entire different language, in order to gain a single feature. Due to this, it makes sense to bring the feature to the language.}
\section{Related Work}
\label{sec:intro_related_work}
This section describes related work within the areas of \ac{STM}. In order to learn from the approaches taken by others as well as achieving a better understanding of the subject a number of papers, articles, and other research material of relevance has been read. This section describes a selection of relevant literature on the subject. The focus has been on identifying different strategies for language integration of \ac{STM} as well as different \ac{STM} implementation strategies.

\subsection{Composable Memory Transactions}
In \cite{harris2005composable} Harris et al. describe their work with integrating \ac{STM} into Haskell. Haskell is extended with support for an \bscode{atomic} function which takes an \ac{STM} action as input and produces an \ac{IO} action as output\cite[p. 51]{harris2005composable}. Evaluating the IO action executes the transaction defined by the atomic function. The Haskell setting allowed the authors to divide the world into \ac{STM} actions and \ac{IO} actions\cite[p. 51]{harris2005composable}, effectively disallowing \ac{IO} actions within transactions as well as only allowing \ac{STM} actions to be performed inside memory transactions. The \ac{STM} system is implemented as a C library integrated into the Haskell runtime system. The Haskell constructs utilize this library to execute transactions\cite[p. 56]{harris2005composable}. Furthermore, the authors contribute with a description and implementation of \ac{STM} constructs for conditional synchronization. The \bscode{retry} statement allows a transaction to block until some condition is met at which point the transaction is aborted and retried\cite[p. 52]{harris2005composable}. The \bscode{orElse} can be used in combination with the retry statement to specify transactional alternatives to be executed in case the previous alternatives encounter a retry\cite[p. 52]{harris2005composable}.

\subsection{\ac{STM} for Dynamic-sized Data Structures}
In \cite{herlihy2003software} Herlihy et al. describes the \ac{DSTM} system. \ac{DSTM} is a library based \ac{STM} system aimed at the C++ and Java programming languages\cite{herlihy2003software}[p. 92]. \ac{DSTM} uses transactional objects which encapsulate regular objects and provide \ac{STM} based access and synchronization\cite{herlihy2003software}[p. 9]. Each transactional object contains a record of its current value, old value and a reference to the transaction which created the record\cite{herlihy2003software}[p. 95]. \ac{CAS} is employed to atomically update the state of a transactional object\cite{herlihy2003software}[p. 96]. \ac{DSTM} is an obstruction-free\cite{herlihy2003obstruction} \ac{STM} system. Obstruction-freedom guarantees that any thread which runs for long enough without encountering a synchronization conflict makes progress\cite{herlihy2003obstruction}[p. 1]. Unlike stronger progress guarantees such as lock-freedom and wait-freedom, obstruction-freedom does not prevent livelock\cite[p. 47]{harris2010transactional}. As a result \ac{DSTM} employs a contention manager to ensure progress in practice\cite{herlihy2003software}[p. 93]. An extended version of \ac{DSTM} called DSTM2 is presented in \cite{herlihy2006flexible}. Here the authors focused on creating a simple and flexible \acsu{API} for the \ac{STM} library.

\subsection{Language Support for Lightweight Transactions}
In \cite{harris2003language} the authors describe how they integrated \ac{STM} into the Java programming language by modifying both the compiler\cite[p. 4]{harris2003language} and virtual machine\cite[p. 9]{harris2003language}. The authors design, implement and performance test an obstruction free \ac{STM} system. The \ac{STM} system uses a non-blocking implementation, and thus guarantees the absence of deadlocks and priority inversion. Additionally, non-conflicting executions are executed concurrently. It uses per object ownership records to track the objects version number as well as which transaction currently owns the object\cite[p. 6]{harris2003language}. Transaction descriptors are employed in order to keep track of the read and write operations performed by a transaction. Transactions are committed  using \ac{CAS} in order to ensure atomicity\cite[p. 7]{harris2003language}. The performance tests show how the \ac{STM} system scales almost as well or better than locks, when the amount of cores available is increased\cite[p. 12]{harris2003language}. The test cases where performed on a ConcurrentHashmap, as opposed to a entire system.

\subsection{Transactional Locking \rom{2}}
In \cite{dice2006transactional} the TL\rom{2} \ac{STM} system designed at Sun Microsystems Laboratories by Dice et al is described. While many of the other \ac{STM} systems described here adopt an obstruction-free approach to implementing \ac{STM}, TL\rom{2} uses commit time locking\cite[p. 199]{dice2006transactional}. A transaction explicitly records its read and write operations in a read and write set\cite[p. 198]{dice2006transactional}. Instead of writing directly to memory, all writes are written to the write set. When a transaction is about to commit it acquires the lock on each object in the write set and writes the values contained in the write set to the actual memory locations before releasing the locks\cite[p. 200]{dice2006transactional}. This corresponds to a two phase locking scheme\cite[p. 455]{tanenbaum2008modern}. A global version clock is used to verify that transactions are executed in isolation\cite[p. 201]{dice2006transactional}. As a transaction starts it reads the current value of the global version clock, storing it locally so it can be used for validation. As a transaction is about to commit it validates its read set by, for every object in the read set, comparing the locally stored read stamp with the objects associated write stamps\cite[p. 200]{dice2006transactional}. If any write stamps are higher than the locally stored read stamp, a conflict has occurred and the transaction must abort and restart. 

\subsection{A (Brief) Retrospective on Transactional Memory}
Inside Microsoft, a group of architects and researchers led an incubation project. Joe Duffy, now director of the Compiler and Language Platform group at Microsoft, gives a retrospective view on their work in \cite{duffy2010stmnet}. Their goal was to provide a first class \ac{STM} implementation, i.e. integrated in the language, supported by \ac{JIT}, \ac{GC}, compiler and debugger. Their overall strategy was to use a version number for optimistic reads, and a lock for writes. Initially they chose weak atomicity and update in-place, but realized that this approach suffered from privatization issues\toby{fortæller vi et sted hvad det er?}, breaking the isolation. They settled on a write on-commit approach, but suffered from false sharing\toby{hvad er false sharing?} between transactional and non-transactional code, due to loss of granularity which was reduced to object level. They chose unbounded transactions, to provide a broader appeal. They would rely on compiler optimization through static analysis to remove unnecessary barriers as well as finding violations of the isolation introduced by the programmer. They identified \ac{STM} as a systemic and platform wide technology shift, just like generics. Having a platform wide change, requires careful integration with existing language features, in order to preserve the orthogonality. They identified several critical operations, that would cause trouble if permitted inside a transaction since their actions are non-reversible per default, e.g. allocation of finalize objects, \ac{IO} calls, GUI operations, P/Invokes to Win32, library calls and use of locks. This ultimately led to the realization that not all problems are transactional, and very little .NET code could actually run inside the transaction, but computations performed solely in memory. This combined with the privatization issue and several minor but continuous arising problems, caused Joe Duffy to state, that the research area of \ac{STM} was, as of January 2010, not mature enough, and thus STM.NET never made it outside of the incubation project.

\section{Scope}\label{sec:scope}
%\andreas[inline]{We are not solving the side-effect problem}
%\andreas[inline]{Bringing the concurrency construct STM into a C\#}
%\kasper[inline]{No coreclr}
%\kasper[inline]{Focus on the C\# part of Roslyn}
\ac{STM} has been an active area of research for almost 20 years\cite{shavit1997software}. While the research has come far from the initial proposal of statically sized memory transactions, the area still has unsolved problems, including issues with side effects, \ac{IO} and exceptions occurring inside transactions\cite{harris2005exceptions}. While more research in these areas would be of interest to the research community, it is not the focus of this master thesis. Furthermore while the creation of new \ac{STM} algorithms with good performance is of interest to the research community this is not the focus of this thesis. Instead the focus is on the integration of \ac{STM} in the C\# programming language. Specifically C\# 5.0, the most recent version at the time of writing\cite{csharp2013specificaiton}.

%As such we restrict the project to investigating integration of \ac{STM} in the C\# programming language and not on solving these known issues.

The Roslyn compiler project contains compilers for both Visual Basic and C\#\cite{roslyn}. As this master thesis focuses on C\# we will restrict any investigation into the Roslyn compiler to focus on the C\# compiler as well as shared functionality of interest. We realize that the Roslyn compiler contains features for the unreleased C\# 6.0, but as these features are uncompleted and not final, they will not be accounted for in our integration.

As of February 2015 Microsoft released the source code for an independent version of its .NET runtime environment \acl{CLR}, called \ac{CLR} Core on Github\cite{coreclr}. While this undoubtedly presents many exiting research opportunities for the area of \ac{STM} as well as other computer science areas, we consider the \ac{CLR} to be out of scope for this thesis. The \ac{CLR} Core is mainly written in C++\cite{coreclr}, a language with which we have only limited experience. Furthermore, the \ac{CLR} Core consists of roughly 2.6 million lines of code\cite{coreclrBlog} making it a complex and time demanding task to gain an understanding of its structure.

\section{Problem Statement}\label{sec:problem_statement}
The goal of this master thesis is to investigate the possibility of supplying language integrated support for \ac{STM} in C\#. To formalize our goal, we have constructed the hypothesis:

\paragraph{Hypothesis} Language integrated and library based \ac{STM} provides a valid alternative to locking when solving known concurrent problems by using C\#.\andreas{Does not solve IO and placing CR}
\andreas{If we change this, we must change Preliminary Knowledge and Evaluation}
\andreas[inline]{Clarifying whether language integration of \ac{STM} provides any advantages over library based \ac{STM} in terms of usability.}
\andreas[inline]{Why did we choose the evaluation method}

In order to validate the hypothesis, a language integrated \ac{STM} system for C\#, called \stmname, will be designed, implemented and integrated\andreas{Is this sufficient, or should we expand and reference to chapters?}. In order to achieve this goal a \ac{STM} library for the .NET platform will be created. Using both of these implementations a number of representative concurrent problems will be implemented. These implementations will be compared to equivalent lock based implementations by quantifying their characteristics in regards to concurrency, using an extended version of the characteristics defined in \cite[p. 16-21]{dpt907e14trending}.

\subsection{Problem Statement Questions}
In order to structure our investigation we have extracted a number of problem statement questions. The questions are based on findings from the theory investigated in our previous study\cite{dpt907e14trending}, investigation of related work, exploratory investigations into \ac{STM} implementations, and the Roslyn compiler. These questions will serve as a guideline for investigating the integration of \ac{STM} into C\#, as well as being used to conclude upon the project in \bsref{chap:conclusion}.

\begin{enumerate}
\item What features should an \ac{STM} system for C\# contain?
\item What problems exist in integrating \ac{STM} in C\#?
\item What different implementation strategies exist for \ac{STM}?
\item How is the Roslyn compiler structured?
\item How can the Roslyn compiler be utilized to integrate \ac{STM} into the C\# language?
\item How does the characteristics differ when using locks or our \ac{STM} in C\#?
\end{enumerate}

Answering these questions along with supplying a prototype implementation will build the foundation for evaluating the hypothesis.

\section{Evaluation Approach}\label{sec:eval_approach}
This section describes how the evaluation is conducted.
\subsection{Selected Problems}
As described in \bsref{sec:problem_statement} the evaluation is conducted by comparing the characteristics of each approach based on a number of concurrent implementations. For this purpose three concurrency problems have been selected:
\begin{enumerate}
\item The Dining Philosophers (\bsref{app:dining_phil})
\item The Santa Claus Problem (\bsref{app:santa})
\item A Concurrent Hashmap (\bsref{app:hashmap})
\end{enumerate}
The Dining Philosophers problem represents a well known concurrency problem which highlights some of the pitfalls associated with conditional synchronization of threads. The Santa Claus problem encompasses a high degree of modeling and requires complex synchronization, e.g. allowing only a predefined number of threads to enter a critical region at a time and waiting on one of multiple conditions. Employing this problem helps investigate which advantages \ac{STM} provides over locking in such scenarios. Finally a concurrent hashmap implementation represents a real world problem, e.g. it can be used in a compiler to maintain a symbol table\bsref{cormen2009introduction}, which benefits from fine grained synchronization and is available in a number of languages, including C\#. Together these problems provide a varied perspective by exerting different aspects of each approach e.g. waiting on one of multiple conditions and fine grained synchronization.

\subsection{Evaluation Of Characteristics}
For each of the selected problems an implementation will be created using locking, library based \ac{STM} and \stmname. Based on these implementations each concurrency approach will be evaluated according to an extended version of the characteristics defined in our previous work\cite[p. 15-21]{dpt907e14trending}. These characteristics are a combination of general characteristics for concurrency models such as pessimistic or optimistic concurrency, as well as characteristics such as simplicity and readability which have been used to evaluate programming languages\cite[p. 7]{sebestaProLang}. The characteristics are extended with the characteristics Data Types and Syntax Design. The reason being, that characteristics are now used to evaluate concrete implementations as opposed to concurrency models in our previous work\cite{dpt907e14trending}.

While each of these characteristics range from one extreme to another, e.g. high or low readability, a concurrency approach may not reside at one of these extremes. Therefore each concurrency approach will be given a placement on the spectrum of each characteristic, based on the findings of the evaluation. In order to visualize this placement a scale similar to the one presented in \bsref{fig:evel_example} is employed. Here \bscode{X} and \bscode{Y} represent the two extremes of the spectrum while the indicators represent the placement of each of the concurrency approaches on the spectrum. As an example \bscode{X} and \bscode{Y} could be low and high writability, \bsref{fig:evel_example} then shows that each of the concurrency approaches resides more towards the high writability end of the spectrum.
\begin{figure}[ht!]
\centering
\includegraphics[scale=0.5]{\rootpath/worksheets/evaluation/figures/eval_example}
\caption{Example of characteristic evaluation scale}\label{fig:evel_example}
\end{figure}
As the evaluation of the characteristics is subjective placing each of the concurrency approaches on the spectrum of the characteristic under evaluation allows for comparison of the findings.

\subsection{Implementations}
To ensure common ground between all implementations of a particular problem, a set of requirements detailing common factors, which must be true for all implementations of a particular problem, has been created. The source is located in \bsref{}\andreas{Source!} together with a description of the chosen approach. Common for all the solutions is, that they must solve the problem by using a concurrent approach and utilizing their individual strengths.

The Dining Philosophers implementations must:
\begin{itemize}
	\item Encompass two 100 milliseconds thread sleeps. One to simulate the act of eating, forcing the locks to be held while a philosophers finishes her meal, and one upon completing an attempt to eat simulating sleeping a period of time before attempting to eat again. The amount of sleep is only of importance in regards to performance testing, which is why an arbitrary number of 100 has been picked, it is only done to provide consistency between the implementations.
\end{itemize}

The Santa Claus problem implementations must:
\begin{itemize}
	\item Adhere to the requirements defined in \bsref{app:santa}.
	\item Utilize advantageous features of the concurrency approach.
\end{itemize}

The Concurrent Hashmap implementations must:
\begin{itemize}
	\item Encompass fined grained synchronization, allowing multiple threads to operate on the hashmap simultaneously.
\end{itemize}
The implementations in their full length can be found in the appendix.\toby{indsæt et reference når de indsætts}
\worksheetend