\makeatletter \@ifundefined{rootpath}{\input{../../setup/preamble.tex}}\makeatother
\worksheetstart{Requirements for \stmname}{0}{Februar 10, 2015}{}{../../}
This chapter describes the requirements for the \ac{STM} system powering \stmname. The requirements are based on known \ac{STM} constructs as well as the criteria for which the final system must be evaluated. The requirements will be used when designing and integrating the \ac{STM} features into C\#, as discussion in \bsref{chap:stm_design}. Furthermore, the requirements will used when the \ac{STM} system must be implemented, as discussed in \bsref{chap:implementation}.
\label{sec:stm_requirements}
\kasper[inline]{Refer to work last semester}
\kasper[inline]{Be able to pass atomic int -> int parameter including ref/out}
\toby[i]{Describe what we will describe in the sections in this chapter}

\section{Tracking Granularity}\label{sec:tracking}
A variable tracked by the \ac{STM} system can be tracked with different granularity: by tracking assignments to the variable, or tracking changes to the object referenced by the variable. These different approaches affect the semantics of the \ac{STM} system, and will be discussed in the subsequent sections.

\subsection{Tracking of Variables} 
Tracking changes to the variable directly limits the effect of the \ac{STM} system only to the variable, and not internal changes inside of the referenced object. This is the approach used in the \ac{STM} system in Clojure\cite{clojureConcurrent}. It offers a simple mental model for the programmer, as only changes visible in the transaction scope will be provided with the transactional guarantees of atomicity and isolation.  \bsref{lst:tracking_variable} shows an example, where the field \bscode{\_car} is assigned to a new modified version of a car on line 11. This assignment is tracked by the \ac{STM} system, as opposed to the example in \bsref{lst:tracking_object}, where the internals of the object are not tracked, when a method with a side-effect is called on line 12. The discussion of side-effects is expanded in \bsref{sec:side-effects}.

This approach can be used in combination with code where the internals are available, such as libraries shipped as binaries, but changes to their internals will not be traced, and therefore the reusability is limited. Should the programmer want transactional support for the side-effects, the internals must be rewritten using the \ac{STM} system.

%
\begin{lstlisting}[label=lst:tracking_variable,
  float,
  caption={Tracking Assignment to Variables},
  language=Java,  
  showspaces=false,
  showtabs=false,
  breaklines=true,
  showstringspaces=false,
  breakatwhitespace=true,
  commentstyle=\color{greencomments},
  keywordstyle=\color{bluekeywords},
  stringstyle=\color{redstrings},
  morekeywords={atomic, retry, orElse, var, get, set, sealed}]  % Start your code-block
  
  sealed class Car {
    private const int _kmDriven;
    public Car(int km) { _kmDriven = km; }
    public Car Drive (int km) { 
      return new Car(_kmDriven + km); 
    }
  }
  
  ...
  atomic {
    _car = _car.drive(25);
  }
\end{lstlisting}
%
\subsection{Tracking of Objects}
Tracking changes to the entire object allows the \ac{STM} system to track side-effects causing changes to the internals of the object. This is useful to reverse a side-effect. E.g. when adding an object to a collection inside a transaction, and the transaction is aborted, the collection should be left in a state not containing the object. Tracking the objects internals reveals an interesting dilemma should the object itself contain other objects. How deep in the object structure should the \ac{STM} system track changes? 

In \cite{herlihy2003software} Herlihy et al. present a library based \ac{STM} called \ac{DSTM}. It uses an approach, where the programmer explicitly has implement a interface allowing the \ac{STM} system to create a shallow copy of the a object. The \ac{STM} systems returns a copy of a variables value to which the transaction can apply its changes.  As the copy is shallow, the internals could reference objects shared with the original, and side-effects on these objects will affect the original, thus be able to break isolation unintentionally. If the programmer want a deeper tracking, she must design the internals using the \ac{STM} system. The example in \bsref{lst:tracking_object} demonstrates a side-effect on \bscode{\_car} being tracked, as \bscode{\_kmDriven} is a value type, whereas the side-effect on \bscode{\_engine} is not, since it is a reference type, thus the reference points to an object which is not tracked by the \ac{STM} system.

In \cite{harris2003language} Harris et al. present an \ac{STM} system which tracks changes throughout the whole object structure, e.g. the objects referenced by the object etc. Changes to objects are buffered in a log, and written to the original if the transaction commits. This deep  traceability is enabled by having a part of the \ac{STM} system in the \ac{JIT} compiler, as the entire structure is known at that level, even if an object is from a compiled library. This approach ensures isolation, but requires modifications to the \ac{JIT} compiler. In the example of \bsref{lst:tracking_object}, both \bscode{\_kmDriven} and \bscode{\_engine} will be tracked by the \ac{STM} system.
%
\begin{lstlisting}[label=lst:tracking_object,
  float,
  caption={Tracking Changes to Object},
  language=Java,  
  showspaces=false,
  showtabs=false,
  breaklines=true,
  showstringspaces=false,
  breakatwhitespace=true,
  commentstyle=\color{greencomments},
  keywordstyle=\color{bluekeywords},
  stringstyle=\color{redstrings},
  morekeywords={atomic, retry, orElse, var, get, set}]  % Start your code-block
  
  class Car {
    private int _kmDriven;
    private Engine _engine;
    public void Drive (int km) { 
      _kmDriven += km; 
      _engine.Degrade(km);
    }
  }
  
  ...
  atomic {
    _car.Drive(25);
  }
\end{lstlisting}
%
\subsection{Choice of Tracking Granularity}
As deep tracking of objects requires changes to the \ac{JIT} compiler, and we have restricted ourselves from changes to \ac{CLR} core in \bsref{sec:scope}, this is not feasible. This leaves the option of tracking changes of the object, but not the objects internals, or only tracking the variable. If we cannot provide transactional support for all side-effects, but only those one level deep, the support will seem inconsistent from the programmers point of view. We therefore choose only to track the variable, since it provides a consistent, simple mental model for the programmer, displaying exactly what is tracked by the \ac{STM} system. The consequence of this choice is, that the side-effects cannot be tracked automatically, thus laying a potential burden on the programmer. 

\section{Transactions \& Variables}
\label{subsec:rec_transactions_variables}
%Trans keyword
%Atomic block
%Read outside transactions
%No write outside transactions
As described in \bsref{sec:stm_keyconcepts_example} an \ac{STM} system must offer some way of defining a transaction scope. As \stmnamesp is a language integrated \ac{STM} system, C\# must be extended with syntax for specifying a transaction scope.\andreas{More on the transaction block?}

An \ac{STM} system abstracts away many of the details of how synchronization is achieved. Simply applying transactions over a number of C\# variables provides a high level of abstraction but also hides the impact of synchronization. Having the programmer specify what variables should be tracked by the \ac{STM} system ensures that she is aware of what variables the \ac{STM} system is managing. Explicitly reasoning about what variables are to be synchronized can assist the programmer in gauging the performance of a transaction as well as better understanding the execution of transactions, both areas which usability studies\cite{rossbach2010transactional}\cite{pankratius2009does} have found to be problematic for some programmers. As with defining transaction scopes, C\# must be extended with syntax for marking variables for synchronization inside transactions.  We define a variable marked for use in transactions a transactional variable. A transactional variable must function similarly to a \bscode{volatile} variable. That is, the language must treat a transactional variable as any other variable of the same type when is it utilized. Thus a transactional variable can be passed as an argument into a method with a non-transaction parameter. Just like a \bscode{volatile} variable, a transactional variable must be treated differently than normal variables by the compiler. These differences must however largely be unnoticeable, in terms of usability, by the program.

Applicability is key when evaluating \stmname, by comparing it to equivalent lock based implementations as described in \bsref{sec:problem_statement}. \stmnamesp must not be overly restrictive as this will limit its applicability. Due to this, the \ac{STM} system must allow reads from transactional variables to occur both inside and outside transactions. For writes to transactional variables a choice exists between allowing and disallowing writes from outside transactions. Disallowing writes from outside transactions will ensure that non-transactional access can not interfere with transactional access but will hamper the usability of the \ac{STM} system. Allowing writes from outside transactions increases the complexity of the implementation, as any conflicts such writes create must be detected and resolved by the \ac{STM} system. As allowing writes from outside transactions provides the best usability \stmnamesp must provide support for such writes in addition to writes from inside transactions. This requirement is closely related to the choice between strong and weak atomicity discussed in \bsref{sec:design_strong_weak_atomicity}.

\section{Strong or Weak Atomicity}
\label{sec:design_strong_weak_atomicity}
The atomicity guarantee provided by \ac{STM} systems varies, depending on the semantics provided. In \cite{blundell2006subtleties} Blundell et al. define two levels of atomicity:
%
\begin{defn}\label{def:strong_atomicity}
\emph{[...] strong atomicity to be a semantics in which transactions execute atomically
with respect to both other transactions and non-transactional code.}
\end{defn}
%
\begin{defn}\label{def:weak_atomicity}
\emph{[...] weak atomicity to be a semantics in which transactions are atomic only with respect to other transactions.}
\end{defn}

Strong atomicity provides non-interference and containment between transactional and non-transactional code, whereas weak atomicity does not. As an example, using the \bscode{Car} class defined in \bsref{lst:atomicity}, having the \bscode{KmDriven} setter called from one thread, while another thread is calling the \bscode{Drive} method, strong and weak atomicity yields different results. Under strong atomicity, all changes made inside the \bscode{atomic} block at line 12, are isolated from non-transactional code. Additionally, changes made from the setter are isolated from inside the atomic block. The result is, that if the setter is called in the middle of the \bscode{Drive} method, there will be a conflict which must be resolved. 

If only weak atomicity is guaranteed, given the same scenario, the change made through the setter, would be visible inside the atomic block. Thus shared memory between transaction and non-transactional code can lead to race conditions.
%

\begin{lstlisting}[label=lst:atomicity,
  caption={Level of Atomicity},
  language=Java,  
  showspaces=false,
  showtabs=false,
  breaklines=true,
  showstringspaces=false,
  breakatwhitespace=true,
  commentstyle=\color{greencomments},
  keywordstyle=\color{bluekeywords},
  stringstyle=\color{redstrings},
  morekeywords={atomic, retry, orElse, var, get, set}]  % Start your code-block

  class Car {
    private int _kmDriven;
    public int KmDriven {
      get {
        return _kmDriven;
      }
      set {
        _kmDriven = value;
      }
    }
    public void Drive (int km) {
      atomic {
        _kmDriven += km;
      }
    }
  }
\end{lstlisting}

\subsection{Issues with Atomicity Levels}
In \cite[p. 30-35]{harris2010transactional} Harris et al. summarizes a collection of issues related to the different levels of atomicity. The collection is non-exhaustive, but based on a wide selection of research. The consequence of race conditions can be either:
\begin{itemize}
	\item Non-repeatable read - if a transaction cannot repeat reading the value of a variable due to changes from non-transactional code in between the readings.
	\item Intermediate lost update - if a write occurs in the middle of a read-modify-write series done by a transaction, the non-transactional write will be lost, as it comes after the transaction has read the value.
	\item Intermediate dirty read - if eager updating\cite[p. 53]{dpt907e14trending} is used, a non-transactional read can see an intermediate value written by a transaction. This transaction might be aborted, leaving the non-transaction with a dirty read.
\end{itemize}
The second case is exactly the case described above in \bsref{lst:atomicity}, where weak atomicity led to the risk of race conditions between transactional and non-transactional code. 

Another issue of using weak atomicity, is known as the privatization problem. If only one thread can access a variable, the need for tracking it through the \ac{STM} system ceases, and so does the overhead. It is therefore desirable, to privatize a previously shared variable when doing intensive work, that does not need to be shared across threads. A technique used for privatizing a variable, \bscode{x}, is to use another variable as a shared marker \bscode{priv}, which indicates whether or not the \bscode{x} is private. This is demonstrated in \bsref{lst:privatization}. Intuitively one would believe, that if \bscode{Thread1} wants to privatize \bscode{x}, it can mark \bscode{priv} in a transaction, and after the transaction ends assume that \bscode{x} is now private. This however is false, since \bscode{Thread2} could read \bscode{priv} and assign to \bscode{x}, after which \bscode{Thread1} executes setting the values of \bscode{priv} and \bscode{x}, causing the transaction of \bscode{Thread1} to abort and rollback. During the rollback the value of \bscode{x} is restored to the value it had when \bscode{Thread2} wrote to it, causing \bscode{Thread1}'s write to \bscode{x}, on line 5, outside of the transaction to be overwritten, and lost. This example is under the assumption of weak atomicity, commit-time conflict detection, and in place updating\cite[p. 34]{harris2010transactional}. 

\begin{lstlisting}[label=lst:privatization,
  caption={Privatization issue},
  language=Java,  
  showspaces=false,
  showtabs=false,
  breaklines=true,
  showstringspaces=false,
  breakatwhitespace=true,
  commentstyle=\color{greencomments},
  keywordstyle=\color{bluekeywords},
  stringstyle=\color{redstrings},
  morekeywords={atomic, retry, orElse, var, get, set}]  % Start your code-block

  // Thread 1
  atomic {
    priv = true;
    }
  x = 100;
  
  // Thread 2
  atomic {
    if (!priv) {
      x = 200;
    }
  }
\end{lstlisting}

\subsection{Choice of Atomicity Level}
All the issues listed above, are related to weak atomicity, and are not present under strong atomicity. Despite of this advantage of strong atomicity, its shortcomings must be concidered before choosing. The overhead of guaranteeing atomicity between transactional and non-transactional code can occur a considerable cost\cite{spear2007privatization}. In \cite{spear2007privatization} Spear et al. proposes four contracts, under which privatization can be guaranteed under some conditions. Strong atomicity is ranked as the least restrictive, but comes with a considerable cost. Although the performance is not optimal, Hindman and Grossman shows in \cite{hindman2006atomicity} that strong atomicity with good performance is achievable by source-to-source compiling with optimizations through static analysis.

As as described in \bsref{sec:problem_statement}, the goal of this project is to validate whether \ac{STM} is a valid alternative to locks by examining their characteristics, and not by performance. Therefore strong atomicity in combination with marked transactional variables is chosen for \stmname. 

\section{Side-effects}\label{sec:side-effects}
Side-effects in methods are a common idiom in C\#, and come in different shapes and form. We categorize side-effects as an in-memory side effects, exceptions or irreversible action. In this section, we will discuss the design requirements for handling the different types of side-effects in \stmname.

\subsection{In-memory Side-effects}
Side-effects in memory is done by modifying state through references to variables outside of the method scope, instead of only returning a new value. An example is \bsref{lst:tracking_object} where the \bscode{Drive} method updates the field \bscode{\_kmDriven} and invokes a method call on \bscode{\_engine}, potentially causing another side-effect. As discussed in \bsref{sec:tracking}, we only track the variable, and not changes to the internals of it. As a consequence, a side-effect such as the one in \bsref{lst:tracking_object} will persist through an aborted transaction. To remedy this, we promote the use of immutable objects, such as the design of the \bscode{Car} class in \bsref{lst:tracking_variable}. This design avoid side-effects, and changes to the object will return a new value, which will be tracked if assigned to the variable.

The immutable approach suites \ac{STM} well, as it is free of side-effects. Additionally it is a less error prone and secure design approach, than mutable objects\cite[p. 73]{bloch2008effective}. Microsoft have an official immutable collection package\footnote{\url{https://www.nuget.org/packages/Microsoft.Bcl.Immutable}}, and is therefore giving first class support for immutability. They also recommend for immutability for ``[...] small classes or structs that just encapsulate a set of values (data) and have little or no behaviors''\footnote{\url{https://msdn.microsoft.com/en-us/library/bb384054.aspx}}. For designing immutable objects, we refer to Bloch's Effective Java\cite[p. 73-80]{bloch2008effective}.

\subsection{Exceptions}
There exist different approaches on how to handle exceptions happening inside a transaction. We discuss these approaches in-depth in our prior study\cite[p. 50-51]{dpt907e14trending}. For \stmnamesp we want to follow programmers intuition on how exceptions work in non-transactional code in transactional code. Therefore, transactions will not be used as a recovery mechanism as proposed by Tim Harris et al. in \cite{harris2005exceptions}, but instead the exception will be propagated if and only if the transaction is committed. Keeping the purpose of transactions only to synchronization keeps it simpler. This way, the programmer will only receive exceptions of code that actually takes effect, and will be able to recover in the usual fashion by catching exceptions. Consequently, the ability to use transactions to recover from exceptions will be lost, but as we want to provide an alternative to locking, as presented in \bsref{sec:problem_statement}, this benefit is negligible.

\subsection{Irreversible Actions}\label{subsec:irreversible}
Effects such as \ac{IO} performed on disk or network, native calls, or GUI operations are not reversible. This makes them unsuitable for use in transactions, since their effect cannot be undone should the transaction be aborted. We expand on this problem in our prior study\cite[p. 51-52]{dpt907e14trending}. In \cite{duffy2010stmnet} Duffy proposes using a well known strategy from the transaction theory\cite{reuter1993transaction}, having the programmer supply on-commit and on-rollback actions to perform or compensate for the irreversible action. In \cite{harris2005exceptions} Harris et al. proposes that \ac{IO} libraries should implement an interface, allowing the \ac{STM} to do callbacks when the transaction is committed, allowing the effect to be buffered until then. These solutions either burden the programmer using \ac{STM}, or the library designer that must implement a special interface.

While we recognize the potential in the presented solutions, the focus of \stmnamesp is limited from solving these well known problems with irreversible actions in \ac{STM}, as presented in \bsref{sec:scope}. Due to this, we do not give any guarantee of the effect of using  irreversible actions in transactions, and it is thus discouraged.

\section{Conditional Synchronization}
\label{sec:req_conditional}
To be a valid alternative to locks in C\#, an \ac{STM} system must be applicable to the same use cases as locks. This requires support for conditional synchronization so that \ac{STM} can be employed in well known scenarios such as shared buffers and other producer consumer setups\cite[p. 128]{tanenbaum2008modern}. \bsref{chap:stm_key_concepts} discusses the \bscode{retry} and \bscode{orElse} constructs proposed in \cite{harris2005composable} for conditional synchronization and composition of alternatives. Supporting such constructs in C\# will increase the applicability of the \ac{STM} system.

Our previous work in \cite{dpt907e14trending} includes an implementation of the $k$-means clustering algorithm\cite[p. 451]{dataminingconceptsandtechniques} in the functional programming language Clojure. Clojure contains a language integrated \ac{STM} implementation which does not support constructs such as \bscode{retry} and \bscode{orElse}. As a result the implementation requires the use of condition variables and busy waiting in scenarios where the \bscode{retry} construct could have been employed\cite{duffy2010stmnet}. Supplying \bscode{retry} and \bscode{orElse} constructs in C\# will allow for simpler conditional synchronization without the need for busy waiting, thereby increasing the simplicity, level of abstraction and writability in such scenarios. 

A disadvantage of providing the \bscode{retry} and \bscode{orElse} constructs is, that the complexity of the \ac{STM} system is increased, and is no longer as simple as if only the construct was \bscode{atomic} transaction scope. However, as the \bscode{retry} and \bscode{orElse} constructs are optional, this is disadvantage is negligible. We therefore choose to include the conditional synchronization constructs.

\section{Nesting}
\label{sec:stm_req_nesting}
The traditional \acl{TL} approach to concurrency has issues with composability due to the threat of deadlocks when composing lock based code\cite[p. 58]{sutter2005software}. \ac{STM} attempts to mitigate these issues by removing the threat of deadlocks, and allowing transactions to nest. Nesting can occur both lexically and dynamically\cite[p. 1]{kumar2011hparstm}\cite[p. 42]{harris2010transactional}\cite[p. 2081]{herlihy2011tm}. 

%\bsref{lst:stm_nested_transactions} shows an example of lexically nested transaction while \bsref{lst:stm_nested_transactions_real} shows an example of dynamically nested transactions. Here the withdraw and deposit methods on the accounts are themselves defined using transactions.
%
%\begin{lstlisting}[label=lst:stm_nested_transactions,
%  caption={Lexically nested transactions},
%  language=Java,  
%  showspaces=false,
%  showtabs=false,
%  breaklines=true,
%  showstringspaces=false,
%  breakatwhitespace=true,
%  commentstyle=\color{greencomments},
%  keywordstyle=\color{bluekeywords},
%  stringstyle=\color{redstrings},
%  morekeywords={atomic, retry, orElse, var}]  % Start your code-block
%
%	atomic{
%		x = 7;
%		atomic{
%			y = 12;		
%		}
%	}
%       
%\end{lstlisting}
%
%\begin{lstlisting}[label=lst:stm_nested_transactions_real,
%  caption={Dynamically nested transactions},
%  language=Java,  
%  showspaces=false,
%  showtabs=false,
%  breaklines=true,
%  showstringspaces=false,
%  breakatwhitespace=true,
%  commentstyle=\color{greencomments},
%  keywordstyle=\color{bluekeywords},
%  stringstyle=\color{redstrings},
%  morekeywords={atomic, retry, orElse, var}]  % Start your code-block
%
%	atomic{
%		var amount = 200;
%		account1.withdraw(amount);
%		account2.deposit(amount);
%	}
%       
%\end{lstlisting}

An \ac{STM} system for C\# must support nesting of transactions as this will allow the system to mitigate one of the major caveats associated with lock based concurrency. A more in depth description of the composability problems of the \ac{TL} concurrency model and nesting of transactions can be found in our prior work \cite{dpt907e14trending}.

Different semantics exist for nesting of transactions. These are:  \begin{inparaenum}
\item Flat
\item Open and 
\item Closed\cite[p. 1]{kumar2011hparstm}\cite[p. 42]{harris2010transactional}.
\end{inparaenum}
Flat nesting treats any nested transactions as part of the already executing transaction, meaning that an abort of the nested transaction also aborts the enclosing transaction. Closed nested semantics allows nested transactions to abort independently of the enclosing transaction. Under closed nested semantics, commits by nested transactions only propagate any changes to the enclosing transaction, as opposed to the entire system. Open nesting allows nested transactions to commit even if the enclosing transaction aborts and propagates changes made by nested transactions to the entire system whenever a nested transaction commits.

Flat nesting is the simplest to implement, but closed and especially open nesting allows for higher degrees of concurrency\cite[p. 43]{harris2010transactional}. Considering the simplicity, readability and level of abstraction provided by the different strategies, as well as the degree of concurrency offered, closed nesting is selected for \stmnamesp. In order to improve the flexibility and orthogonality \stmnamesp is required to support both lexical and dynamic nesting.

\section{Opacity}
\label{sec:stm_req_opacity}
Opacity is a correctness criteria requiring transactions to only read consistent data throughout their execution\cite[p. 1]{guerraoui2007opacity}\cite[p. 29]{harris2010transactional}. This means that transactions must not read data which would cause them to abort at a later time. Consequently opacity requires that the value read is consistent when the read occurs, but allows the variable to be changed at some later point by another transaction. Transactions must be aborted when reads can not be guaranteed to be consistent.

By providing opacity programmers do not have to reason about problems that occur as a result of inconsistent reads\cite[p. 28]{harris2010transactional}, thereby simplifying the programming model. As an example, consider \bsref{lst:stm_opacity}.

\begin{lstlisting}[label=lst:stm_opacity,
  caption={Opacity example},
  language=Java,  
  showspaces=false,
  showtabs=false,
  breaklines=true,
  showstringspaces=false,
  breakatwhitespace=true,
  commentstyle=\color{greencomments},
  keywordstyle=\color{bluekeywords},
  stringstyle=\color{redstrings},
  morekeywords={atomic, retry, orelse, var, get, set, using}]  % Start your code-block

  using System.Threading;

  public class Opacity
  {
      private atomic static int X = 10;
      private atomic static int Y = 10;

      public static void Main(string[] args)
      {
          var t1 = new Thread(() =>
          {
              atomic
              {
                X = 20;
                Y = 20;
              }
          });

          var t2 = new Thread(() =>
          {
              atomic
              {
                  var tmpx = X;
                  var tmpy = Y;
                  while (tmpx != tmpy)
                  {
                  }
              }
          });

          t1.Start();
          t2.Start();
      }
  }
\end{lstlisting}
Here we see two transactional variables \bscode{X} and \bscode{Y} defined on lines 5 and 6 as well as the two threads \bscode{t1} and \bscode{t2} defined on lines 10 and 19. \bscode{t1} simply sets the value of \bscode{X} and \bscode{Y} as a transaction. \bscode{t2} enters a transaction in which it reads the values of \bscode{X} and \bscode{Y} entering a loop if the values are not equal. Consider the interleaving shown in \bsref{fig:opacity_interleaving}. The transaction executed by \bscode{t2} reads the value 10 associated with the variable X after which \bscode{t1}'s transaction updates the value both of \bscode{X} and \bscode{Y} to 20. \bscode{t2} reads the value 20 associated with \bscode{Y}. In an \ac{STM} system providing opacity this would not be allowed since the transaction would read inconsistent data. If the \ac{STM} system does not provide opacity the read would be allowed and \bscode{t2} would go into an infinite loop as \bscode{tmpx} and \bscode{tmpy} are not equal. 

\begin{figure}[htbp]
\centering
 \includegraphics[width=0.65\textwidth]{\rootpath/worksheets/stm_requirements/figures/opacity_interleaving} 
 \caption{Opacity interleaving example}
\label{fig:opacity_interleaving}
\end{figure}

As opacity bolster the simplicity of using \ac{STM}, it is required for \stmname. This will positively impact the level of abstraction, readability and writability.

\section{Summary of Requirements}
In the following section, the requirements will be summarized in order to get a clear overview of properties \stmnamesp must have. 

The granularity of tracking \stmnamesp will provide, is on the variable level. That is, the assignments to variables will be tracked, however side-effects to the referenced object will not. Tracking the internals of an object will require the objects field to be marked as transactional variables. A transaction scope can be defined by a syntax extension provided in \stmname. Transactions in \stmnamesp is under the guarantee of strong atomicity, providing isolation between transactional and non-transactional code. As the \ac{STM} system does not track side-effects, use of immutable objects is promoted. Exceptions occurring inside a transaction will be propagated out of the transaction, in if the transaction is able to commit. Thus transactions will only be used for synchronization, not recovery of state. All irreversible actions, as mentioned in \bsref{subsec:irreversible} are discouraged. \stmnamesp must facilitate conditional synchronization, this is done by supplying the \bscode{retry} and \bscode{orElse} constructs. Nesting is allowed in \stmnamesp under closed nesting semantics, which strikes a balance between minimizing aborts and ensuring simple semantics for nested transactions. Lastly, opacity will be required for \stmname, as this correctness criteria will bolster the simplicity of the \ac{STM} system.
\worksheetend