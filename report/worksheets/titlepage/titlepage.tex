\makeatletter \@ifundefined{rootpath}{\input{../../setup/preamble.tex}}\makeatother
%\worksheetstart{Titlepage}{0}{December 31, 2012}{../../}
\begin{titlingpage}
\aautitlepage{%
  \englishprojectinfo{
    Language Integrated STM in C\# Using the Roslyn Compiler - An Alternative to Locking. %title
    %STM Integration in C\# and the Roslyn Compiler. %title
  }{%
    Spring Semester 2015 %project period
  }{%
    dpt109f15  % project group
  }{%
    %list of group members
    Tobias Ugleholdt Hansen\\
    Andreas Pørtner Karlsen\\ 
    Kasper Breinholt Laurberg\\
  }{%
    %list of supervisors
     Lone Leth Thomsen
  }{%
    \today % date of completion
  }%
}{%department and address
  \textbf{Department of Computer Science}\\
  Selma Lagerløfs Vej 300\\
  DK-9220 Aalborg Ø\\
  \href{http://www.cs.aau.dk}{http://www.cs.aau.dk}
}{% the abstract
% Our motivation
% What have we done
% How did we do it
% Our contribution
This master thesis contributes with \stmnamesp -- an integration of \ac{STM} into C\#. Motivated by the increased need for concurrency, \stmname offers an alternative approach for synchronization than locking. \stmname uses transactional blocks to specify critical regions, with orelse blocks for alternative executions, the retry statement for conditional synchronization, and transactional annotation of variables to mark what should be tracked in transactions. The goal was to investigate the usability of \stmname, and compare it to library based \ac{STM} and locking. This was done by using the three approaches to implement solutions to four different concurrency problems, representing different aspects of concurrent programming. These solutions were then analyzed by their characteristics in a qualitative manner. Our evaluation concludes that \stmname is a valid alternative to locking, and provides better usability than library based \ac{STM}. \stmname was build by extending the open source Roslyn C\# compiler. The integration required insight to the Roslyn compiler, which was documented as an additional contribution.
}
\end{titlingpage}