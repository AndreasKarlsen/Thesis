\makeatletter \@ifundefined{rootpath}{\input{../../setup/preamble.tex}}\makeatother
\worksheetstart{STM Design}{0}{Februar 10, 2015}{}{../../}
This chapter describes the process of designing a \ac{STM} system for the C\# programming language. \bsref{sec:stm_requirements} highlights the requirements for the \ac{STM} system while \bsref{sec:stm_design} describes the \ac{STM} systems design. Implementation details are described in \bsref{chap:implementation}.
\label{chap:stm_design}
\section{Requirements}
\label{sec:stm_requirements}

\kasper[inline]{Refer to work last semester}
\subsection{Strong or Weak Atomicity}
\subsection{Conditional Synchronization}
\subsection{Nesting}
The traditional \ac{TL} approach to concurrency has issues with composability due to the threat of deadlocks\cite[p. 58]{sutter2005software} when composing lock based code. \ac{STM} attempts to mitigate this issues by removing the threat of deadlock, and allowing transactions to nest. Nesting can occur both lexically and dynamically\cite[p. 1]{kumar2011hparstm}\cite[p. 42]{harris2010transactional}\cite[p. 2081]{herlihy2011tm}. \bsref{lst:stm_nested_transactions} shows a example a lexically nested transaction while \bsref{lst:stm_nested_transactions_real} shows a example of dynamically nested transactions. Here the withdraw and deposit methods on the accounts are themselves defined using transactions.

\begin{lstlisting}[label=lst:stm_nested_transactions,
  caption={Lexically nested transactions},
  language=Java,  
  showspaces=false,
  showtabs=false,
  breaklines=true,
  showstringspaces=false,
  breakatwhitespace=true,
  commentstyle=\color{greencomments},
  keywordstyle=\color{bluekeywords},
  stringstyle=\color{redstrings},
  morekeywords={atomic, retry, orElse, var}]  % Start your code-block

	atomic{
		x = 7;
		atomic{
			y = 12;		
		}
	}
       
\end{lstlisting}

\begin{lstlisting}[label=lst:stm_nested_transactions_real,
  caption={Dynamically nested transactions},
  language=Java,  
  showspaces=false,
  showtabs=false,
  breaklines=true,
  showstringspaces=false,
  breakatwhitespace=true,
  commentstyle=\color{greencomments},
  keywordstyle=\color{bluekeywords},
  stringstyle=\color{redstrings},
  morekeywords={atomic, retry, orElse, var}]  % Start your code-block

	atomic{
		var amount = 200;
		account1.withdraw(amount);
		account2.deposit(amount)
	}
       
\end{lstlisting}

\ac{STM} system for C\# must support nesting of transaction as this will allow the \ac{STM} to mitigate the before mentioned issues and be a valid alternative to the existing locked based synchronization mechanisms. A more in depth description of the composability problems of the \ac{TL} concurrency model and nesting of transactions can be found in \cite{dpt907e14trending}.


%Flat, open, closed.

\section{Design}
\label{sec:stm_design}
Something with\cite[p. 1]{harris2003language}


%\section{Lock or Obstruction Free}
%\section{Handling Side-effects}
%Immutable collections
%\section{Contention Manager}
%\section{Log or In Place Update}
%\section{Lazy or Eager Update}



\worksheetend