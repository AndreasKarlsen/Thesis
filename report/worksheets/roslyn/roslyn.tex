\makeatletter \@ifundefined{rootpath}{\input{../../setup/preamble.tex}}\makeatother
\worksheetstart{Roslyn}{0}{Februar 10, 2015}{}{../../}
This chapter describes the Microsoft Roslyn project, hereafter referred to as Roslyn. Little literature on Roslyn exists. The main source is \cite{ng2012roslyn}, which is a whitepaper from Microsoft presenting an overview of Roslyn. However the whitepaper's main focus is on the API side of Roslyn, described in \bsref{sec:intro}, and not the internal details of compilation, which are required in order to integrate \ac{STM} into the Roslyn C\# compiler. Beyond the whitepaper, blog and forum posts have been used, but most of these also only describe the API side. The rest of the knowledge described in this chapter is obtained by inspecting and debugging the source code.

The purpose of this chapter is to obtain the knowledge required to modify the C\# compiler with \ac{STM} support, in order to build \stmname. As a result this chapter does not cover the \ac{VB} aspects of Roslyn. The extension of the Roslyn compiler is described in \bsref{chap:roslyn_extension}.

In \bsref{sec:intro} an introduction to the Roslyn project is given. In \bsref{sec:roslyn_archi} the architecture of Roslyn is described. Following this, \bsref{sec:compile_phases} describes the internal details of the compiler phases, information which is valuable in order to select how and where to extend the compiler with \ac{STM} support. Finally in \bsref{sec:syntax_trees}, a detailed description of the syntax trees of the Roslyn compiler is given.%, as we will extend the C\# compiler with \stmnamesp by modifying the syntax tree\toby{som beskrevet i - hvor?}.
\label{chap:roslyn}

%Extra:
	%it was first introduced in October 2011\footnote{\url{http://blogs.msdn.com/b/csharpfaq/archive/2011/10/19/introducing-the-microsoft-roslyn-ctp.aspx}} as a \ac{CTP} and was recently opensource in 2014
	%also they have changed some things since the first roslyn releases, which means that code examples for inspiration have had to be modified in order to be true for the current roslyn implementation
\section{Introduction}\label{sec:intro}
Project Roslyn is Microsoft's initiative of completely rewriting the C\# and \ac{VB} compilers, using their respective managed language. Roslyn was released as open source at the Microsoft Build Conference 2014\cite{csharpBuild}.

Beyond changing the languages the compilers are written in, Roslyn provides a new approach to compiler interaction and usage. Traditionally a compiler is treated as a black box which receives source code as input and produces object files or assemblies as output\cite[p. 3]{ng2012roslyn}. During compilation the compiler builds a deep knowledge of the code, which in traditional compilers is unavailable to the programmer and discarded once the compilation is done. This is where Roslyn differs, as it exposes the code analysis of the compiler by providing an \ac{API}, which allows the programmer to obtain information about the different compilation phases\cite[p. 3]{ng2012roslyn}.

The compiler \acp{API} available are illustrated in \bsref{fig:api_vs_compiler_pipeline} where each \ac{API} corresponds to a phase in the compiler pipeline. In the first phase the source code is turned into tokens and parsed according to the language’s grammar. This phase is exposed through an \ac{API} as a syntax tree. In the second phase declarations, i.e. namespaces and types from code and imported metadata, are analyzed to form named symbols. This phase is exposed as a hierarchical symbol table. In the third phase identifiers in the code are matched to symbols. This phase is exposed as a model which contains the result of the semantic analysis. This model is referred to as a semantic model and exposes methods that answer semantics questions related to the syntax tree for which it is created\cite{ng2012roslyn}. Through the semantic model programmers can obtain information such as:
\begin{itemize}
\item The type of an expression
\item The symbol corresponding to a declaration
\item The target of a method invocation
\end{itemize}
In the last phase, information gathered throughout compilation is used to emit an assembly. This phase is exposed as an \ac{API} that can be used to produce \ac{CIL} bytecode\cite[p. 3-4]{ng2012roslyn}.

\begin{figure}[htbp]
\centering
\includegraphics[width=1\textwidth]{\rootpath/worksheets/roslyn/figures/compiler_pipeline_vs_api}
\caption{Compiler pipeline in contrast to compiler \acp{API}\cite[p. 4]{ng2012roslyn}.}
\label{fig:api_vs_compiler_pipeline}
\end{figure}

Knowledge obtained through the \acp{API} is valuable in order to create tools that analyze and transform C\# or \ac{VB} code. Furthermore Roslyn allows interactive use of the languages using a \ac{REPL}\cite{cSharp2012interactive}, and embedding of  C\# and \ac{VB} in a \ac{DSL}\cite[p. 3]{ng2012roslyn}.

%\section{Roslyn Overview}\label{sec:inside_ros}\toby{Måske slet den her overskrift og ryk de indre sektioner en tak ud (fordi ellers skal syntax tree vel også ind under roslyn overview)}
%In this section we describe the architecture of Roslyn's source code and further elaborate on the compiler phases from \bsref{fig:api_vs_compiler_pipeline}. \toby{mere forklaring(indledning)}
%evt. skriv at den indre arktitektur skal give et overview, som gør det lettere at beskrive hvor de enkelte faser sker i koden i "compile phases"

\section{Roslyn Architecture}\label{sec:roslyn_archi}
\begin{figure}[htbp]
\centering
\includegraphics[width=0.4\textwidth]{\rootpath/worksheets/roslyn/figures/roslyn_solution_overview}
\caption{Overview of projects in Roslyn solution.}
\label{fig:roslyn_solution_overview}
\end{figure}

The Roslyn solution available on github\footnote{\url{https://github.com/dotnet/roslyn}}, forked\footnote{\url{https://github.com/Felorati/roslyn}} on the 9th February 2015, consists of 118 projects which include projects for Visual Studio development, interactive usage of the languages and more as illustrated on \bsref{fig:roslyn_solution_overview}. The \bscode{Compilers} folder contains the source code for the C\# and \ac{VB} compiler, each located in a separate folder. They share common code and functionality located within the \bscode{Core} folder, including code for controlling the overall compilation flow. Both compilers use the same patterns for compilation\cite[09:36-10:36]{campbellDeeperRos}.

\begin{figure}[htbp]
\centering
\includegraphics[width=0.4\textwidth]{\rootpath/worksheets/roslyn/figures/roslyn_csharp_overview}
\caption{Overview of CSharp folder.}
\label{fig:roslyn_csharp_overview}
\end{figure}

The projects contained in the \bscode{CSharp} folder are shown on \bsref{fig:roslyn_csharp_overview}. The \bscode{csc} project is the C\# command line compiler, which is the starting point of a C\# compilation. The \bscode{CSharpCodeAnalysis.Portable} and \bscode{CSharpCodeAnalysis.Desktop} projects contain the C\# code analysis, which is the actual code required for compilation. The rest of the projects in the \bscode{CSharp} folder mainly involve tests for the C\# compiler. The \bscode{Core} folder has a structure similar to that of the \bscode{CSharp} folder, encompassing a  \bscode{CodeAnalysis.Portable} and \bscode{CodeAnalysis.Desktop} project which contain the common core analysis code.

For more information about the architecture, an overview of the Roslyn Compilers call chain can be found in \bsbilagref{app:roslyn_call_chain}.

\section{Compiler Phases}\label{sec:compile_phases}
The C\# compiler builds upon concepts from traditional compiler theory, such as lexing, parsing, declaration processing, semantic analysis and code generation\cite{sebestaProLang}\cite{fischer2009crafting}. Throughout the phases of compilation, traditional concepts such as syntax trees, symbol tables and the visitor pattern\cite[p. 366]{gamma1994design} are also used. This section elaborates on the compiler phases in the compiler pipeline shown in \bsref{fig:api_vs_compiler_pipeline}.

\begin{figure}[htbp]
\centering
\includegraphics[width=0.4\textwidth]{\rootpath/worksheets/roslyn/figures/csharp_codeanalysis_overview}
\caption{Overview of the \bscode{CSharp.CSharpCodeAnalysis.Portable} project.}
\label{fig:roslyn_csharpanalysis_overview}
\end{figure}

\subsection{Initial Phase}
The initial phase of compilation entails initial work, such as parsing the command line arguments and setting up for compilation, described in further detail in \bsbilagref{app:roslyn_call_chain}. This phase is executed sequentially\cite{sadovRoslynPerf}.

\subsection{First Phase}
\label{subsec:roslyn_first_phase}
The first phase involves parsing the source code, which is done in a traditional compiler fashion by lexing source code into tokens and parsing them into a syntax tree, which represents the syntactic structure of the source code. The lexer is implemented using a \bscode{switch} which identifies the type of token to lex, given the first character of the token string. The parser is implemented as a top-down parser using the common recursive descent approach. The parsing phase will check for syntax errors in source code, but does not have enough information to check for semantic errors, such as scope or type errors. The phase is concurrent, as several files may be parsed simultaneously\cite{sadovRoslynPerf}. The code for this phase is mainly located within the \bscode{Parser} and \bscode{Symbols} folder. The syntax tree and its contents are described in more detail in \bsref{sec:syntax_trees}.

\subsection{Second Phase}
\label{subsec:roslyn_second_phase}
The second phase involves creating a \bscode{Compilation} type object, specifically a \bscode{CSharpCompilation} object. A \bscode{Compilation} object contains information necessary for further compilation e.g. all assembly references, compiler options and source code. In the creation of a \bscode{CSharpCompilation} object, a declaration table is created which keeps track of type and namespace declarations in source code\cite{sadovRoslynPerf}. This is done sequentially\cite{sadovRoslynPerf}. Additionally the \bscode{Compilation} object contains a symbol table, which holds all symbols declared in source code or imported assemblies. Each namespace, type, method, property, field, event, parameter and local variable is represented as a symbol and stored in the symbol table\cite[p. 14]{ng2012roslyn}. Each type of symbol has its own symbol class, e.g. \bscode{MethodSymbol}, which derives from the base \bscode{Symbol} class. The symbol table can be accessed by the \bscode{GlobalNamespace} property, as a tree of symbols, rooted by the global namespace symbol. Furthermore a range of methods and properties to obtain symbols also exists. The code for this phase is mainly located within the \bscode{Declarations} and \bscode{Symbols} folders.
%compilation is immutable, like syntax trees

\subsection{Third Phase}
In order to enable semantic analysis, the third phase entails fully binding all symbols, which determines what each symbol actually refers to, e.g. what namespace and overloaded method, a particular method refers to. Any problems with symbol binding, like inheritance loops, will be reported. Binding is done concurrently, however binding members of a type will force base types to be bound also. For symbols not related, they can be bound in any order \cite{sadovRoslynPerf}.

Additionally, the binding phase also creates a bound tree, which is the Roslyn compilers internal tree used for flow analysis and emitting. In \cite{roslynBinder} Anthony D. Green states that they do not want to expose the bound tree through the \ac{API} as:

\bsqoute{It has been a long standing design decision not to expose the bound tree. The shape of the bound nodes is actually pretty fragile compared to the shape of the syntax nodes and can be changed around a lot depending on what scenarios need to be addressed. We might store something somewhere one day for performance and remove it the next for compatibility or vice versa}{Anthony D. Green\cite{roslynBinder}}

Data flow and control flow analysis uses the bound tree to for instance check if statements are reachable, as the C\# specification states that an unreachable statement should produce a warning\cite{gafter2011}. The code for this phase is mainly located in the \bscode{Binder,} \bscode{BoundTree,} and \bscode{FlowAnalysis} folders.

\subsection{Final Phase}
Finally, in the fourth and final phase all information built up so far is emitted as an assembly. Method bodies are compiled and emitted as \ac{CIL} concurrently. However methods within the same type are compiled sequentially in a fixed order, typically lexical. The final emitting to the assembly is done sequentially\cite{sadovRoslynPerf}. The code for this phase is mainly located in the \bscode{CodeGen} and \bscode{Emitter} folders.

\section{Syntax Trees}\label{sec:syntax_trees}
Syntax trees are the primary structure used throughout compilation. Syntax trees in the Roslyn compiler have three key attributes\cite[p. 6]{ng2012roslyn}:
\begin{enumerate}
	\item A syntax tree is a full fidelity representation of source code, which means that everything in the source code is represented as a node in the syntax tree. If programs are invalid, the syntax tree represents these errors in source code by tokens named \bscode{skipped} or \bscode{missing} in the tree.
	\item A syntax tree produced from parsing must be able to be translated back to the original source code. This is referred to as being roundtripable.
	\item Syntax trees are immutable and thread-safe. This enables multiple users to use the same syntax tree in different threads without concurrency issues. As syntax trees are immutable, factory methods exist in order to help create and modify trees. Upon a modification, the factory methods does not copy the entire tree along with the modification, instead the underlying nodes are reused. As a result, trees can be modified fast and with a low memory overhead.
\end{enumerate}

\subsection{Red And Green Trees}
\label{subsec:roslyn_red_green_trees}
The Roslyn team wanted a primary data structure for compilation with the following characteristics\cite{lippert2012redgreen}:
\begin{itemize}
	\item Immutable.
	\item Form of a tree.
	\item Cheap access to parent nodes from child nodes.
	\item The ability to map from a node in the tree to a character offset in the source code.
	\item The ability to reuse most nodes in the original tree when modifying trees.
\end{itemize}
However fitting all those characteristics into a single data structure is problematic\cite{lippert2012redgreen}:
\begin{itemize}
\item One problem is simply constructing a tree node, because both the child and parent are immutable and must have a reference to each other, so it is not possible to create one before the other.
\item Another problem is reusing nodes for other parents when modifying the tree, as nodes are immutable and it is therefore not possible to change the parent of a node.
\item A third problem is inserting a new character into the source code, as the position in source code of all nodes changes after that point. This is makes it problematic to adhere to the characteristic of reusing most nodes when modifying trees, because a modification to source code can change the character offset of many nodes.
%til den sidste, husk at de selv bruger roslyn til at udvikle interne værktøjer, derfor er det vigtigt at når man skrive et nyt tegn i starten af source koden, at de ikke skal bygge træet helt på nyt.
\end{itemize}

Instead the Roslyn compiler uses two types of trees, green trees and red trees, in order to fulfill all their required characteristics.

\paragraph{The green tree} is immutable, has the ability upon modification to reuse most unaffected nodes, has no parent references, is built bottom-up, and does not know the absolute positions of nodes in the source code, only their widths\cite{lippert2012redgreen}.

For the expression ``5*5+5*5'', a typical parse tree is shown in \bsref{fig:normal_syntax_tree}, and a potential green tree is shown in \bsref{fig:green_syntax_deduplication}. As the green tree nodes do not have parent references and positions in source code, sub-trees and nodes can be reused, which results in a more compact tree. Factory methods are used to create new nodes in the tree in order to determine if existing nodes can be reused or new ones must be created. If nodes for a given expression already exist they are reused, otherwise new nodes are created.

However reuse of existing nodes is not guaranteed, as that requires all nodes to be cached, which according to Vladimir Sadov from Microsoft in \cite{sadovRoslynPerf} makes the reuse caches unnecessarily big. The caches are instead a fixed size, where new nodes will replace older ones when the maximum size is reached. Sadov states that it works pretty well because recently accessed nodes are likely to be accessed in the near future. Another trade-off is that they do not reuse non-terminals with more than 3 children, as it gets more expensive and less likely to match, the more children a non-terminal has\cite{sadovRoslynPerf}.

\begin{figure}[htbp]
\centering
\includegraphics[width=\textwidth]{\rootpath/worksheets/roslyn/figures/normal_syntax_tree}
\caption{Typical parse tree of expression.}
\label{fig:normal_syntax_tree}
\end{figure}

\begin{figure}[htbp]
\centering
\includegraphics[width=0.6\textwidth]{\rootpath/worksheets/roslyn/figures/green_syntax_deduplication}
\caption{Green tree of expression, reusing identical sub-trees. Inspired by \cite{sadovRoslynPerf}.}
\label{fig:green_syntax_deduplication}
\end{figure}

\paragraph{The red tree} is an immutable facade built around the green tree. It has parent references and knows the absolute positions of nodes in the source code. However these features prevent nodes from being reused, which means that making modifications to a red tree is expensive. Therefore another approach than building a new red tree upon every modification\cite{lippert2012redgreen} is used.

The red tree is built lazily using a top-down approach, starting from the root of the tree and descending into children. Once the parent of a node is available, all information required to construct a red node is available. The internal data for the node can be obtained from the corresponding green node. Furthermore the absolute position of the node in source code can be computed, as the position of the parent is known, along with the source code width of all children that come before the given node\cite{sadovRoslynPerf}.

So when modifications are made to the source code, an entire new red tree is not computed. Instead the green tree is modified, which is a relatively cheap operation, because most nodes can be reused. In terms of the red tree, a new red node is created as root with 0 as position, null as parent and the root green node as the corresponding green node. The red tree will then only build itself if a user descends into its children and it might only descend into a small fraction of all the nodes in the tree\cite{sadovRoslynPerf}. % i forhold til hvad det er godt til? med ide typing tankegang

\subsection{Contents Of Syntax Trees}
The elements contained within syntax trees are syntax nodes, tokens and trivia. In this section these constructs and some of their properties are described. Additionally the supporting constructs Spans, Kinds and Errors are described.

\subsubsection{Syntax Nodes}
Syntax nodes represent non-terminals of the language grammar, such as declarations, statements and expressions. Each syntax node type is represented as a separate class that derives from the base \bscode{SyntaxNode} class. As syntax nodes are non-terminals, they always have children, either in the form of other syntax nodes or syntax tokens.

In relation to navigating syntax trees, all syntax nodes has\cite[p. 7]{ng2012roslyn}:
\begin{itemize}
\item A parent property to obtain the parent node
\item For each child, a child property to obtain the child
\item A \bscode{ChildNodes} method to return a list of all child nodes
\item Descendant methods i.e. \bscode{DescendantNodes}, \bscode{DescendantTokens} and \bscode{DescendantTrivia}, to obtain a list of all descendant nodes, tokens or trivia for a given node.
\end{itemize}

Additionally, optional children are allowed, which are represented as \bscode{null} if they are not present. For example an \bscode{IfStatementSyntax} syntax node has an optional \bscode{ElseClauseSyntax} syntax node\cite[p. 7]{ng2012roslyn}.

\subsubsection{Syntax Tokens}
Syntax tokens represent terminals of the language grammar, such as keywords, literals and identifiers. As opposed to syntax nodes, which do not have any children.

Different types of syntax tokens do not have a separate class, instead all syntax tokens are represented by a single \bscode{SyntaxToken} type. This means that there is a single structure for all tokens. To get the value of a token, three properties exist: \bscode{Text}, \bscode{ValueText} and \bscode{Value}. The first returns the raw source text as a \bscode{String}, including extra characters such as escape characters. The second returns only the value of the token as a \bscode{String.} The last returns the value as the actual value type e.g. if the token is an integer literal then the property returns the actual integer. To allow different return types, the return type of the last property is \bscode{Object}\cite[p. 7-8]{ng2012roslyn}.

Additionally, for performance reasons the \bscode{SyntaxToken} type is defined as a struct\cite[p. 7]{ng2012roslyn}.

\subsubsection{Syntax Trivia}
Syntax trivia represent parts of source code that can appear between any two tokens, such as whitespace and comments. Syntax trivia is not included as a child node in the tree, but instead associated with a given token. A syntax token holds all the following trivia on the same line, up to the next token. Syntax tokens hold trivia in two collections: \bscode{LeadingTrivia} and \bscode{TrailingTrivia}. The first token holds all leading initial trivia, and the end-of-file token holds the last trailing trivia in source code\cite[p. 8]{ng2012roslyn}.

As trivia are not nodes in the tree, they do not have a \bscode{Parent} property. Instead the associated token for some trivia, can be accessed with the \bscode{Token} property. Additionally, like syntax tokens, trivia are also structs and have only a single \bscode{SyntaxTrivia} type to describe them all.

\subsubsection{Spans}
Every node, token, and trivia knows its position in source code. This is accomplished by the use of a \bscode{TextSpan} struct type. A \bscode{TextSpan} object holds the start position of a node, token, or trivia in source code and a count of characters, both represented as 32-bit integers\cite[p. 8]{ng2012roslyn}.

Every node, token and trivia has two properties to obtain spans: \bscode{Span} and \bscode{FullSpan}. The \bscode{Span} property includes only the span of the node, token or trivia and not any trivia, where the \bscode{FullSpan} property includes the normal span and any leading or trailing trivia. 

\subsubsection{Kinds}
\label{subsubsec:roslyn_kinds}
Every node, token and trivia has an integer \bscode{RawKind} property, used to identify the syntax element type. Each language, C\# or \ac{VB}, contains a \bscode{SyntaxKind} enumeration that contains all the nodes, tokens and trivia in the language. The \bscode{RawKind} property corresponds to an item in the  \bscode{SyntaxKind} enumeration for the specific language. The \bscode{CSharpSyntaxKind} method gets and automatically cast the bscode{RawKind} to an item in the \bscode{CSharpSyntaxKind} enumeration\cite{roslynwikiOverview}\cite[p. 9]{ng2012roslyn}.

Kinds are especially important for tokens and trivia, as they have only a single type, \bscode{SyntaxToken} and \bscode{SyntaxTrivia}. Thus, the only way to identify the particular token or trivia at hand, is by identifying its associated kind.

\subsubsection{Errors}
If programs are invalid as a result of errors in source code, a syntax tree is still produced. These errors are represented as special tokens in the syntax tree, which are added using one of the following techniques\cite[p. 9]{ng2012roslyn}.
\begin{enumerate}
\item Insert a missing token in the syntax tree when the parser scans for a particular token but does not find it. The missing token represents the expected token, but it has an empty span and has a true \bscode{IsMissing} property.
\item Skip tokens until the parser finds a token from where it can continue the parse. The skipped tokens are added as a trivia node with the \bscode{SkippedTokens} kind.
\end{enumerate}

\subsection{Syntax Tree Generation}
\label{subsec:roslyn_syntax_tree_generation}
The nodes, associated factory methods and visitor pattern for the syntax trees are generated based on the contents of the \bscode{Syntax.xml} file located in the \bscode{CSharpCodeAnalysis (Portable)} project. The contents of the file describes information such as the fields and base class for each node in the tree. Whenever the compiler is built, a tool which source code resides in the \bscode{Tools\textbackslash CSharpSyntaxGenerator} project is run. It generates the classes for each node defined in \bscode{Syntax.xml} along with relevant factory methods, properties for getting the value of each field associated with a node, and factory methods for generating an updated version of the node. The tool also generates the visitor pattern implementation along with the required \bscode{accept} methods on each node. Both the red and green tree, described in \bsref{subsec:roslyn_red_green_trees}, are generated during this process.

%EXTRA
%semantic model evt. (tror kun det er i forhold til api)
	%følgende sker direkte i runcore kald
%compilation = source files, assembly references and compiler options. Main: symbol table
	%består af symboler (Symbol klassen)
	
	%måske sker den semantiske analyse i analyzedriver kaldet, eller GetDeclarationDiagnostics()
%semantic model = semantisk analyse, så som type og scope checking,
	%kan obtains udfra en compilation
	%MERE:
	%The symbols referenced at a specific location in source.
	%The resultant type of any expression.
	%All diagnostics, which are errors and warnings.
	%How variables flow in and out of regions of source.
	%The answers to more speculative questions.
	
%selve emitting sker også i runcore
	%data flow analyse, Control flow analysis også en del af emit (refer til deres paper det ligger på gitten omkring data flow)

%***EKSTRAAAA***
%dataflow og controlflow analyse sker også heri (nævn evt.)
%codegen, emitter og flowanalysis folders
%og i core: codegen og emitter (evt. skriv noget om corresponding i core, når jeg beskriver de to i csharp folderen. Eller deres corresponding base classes in core (hvilket ligger i de folere))

%***HUSK:
%også methodcompiler.cs i Compiler folder (sker faktisk inden endelig emitting)
	%bruger visitor pattern
%herefter controlflow pass (visistere bound tree)
%herefter dataflow pass

	
%CSharpCompilation GlobalNamespace

%Binder =  A Binder converts names in to symbols and syntax nodes into bound trees. It is context dependent, relative to a location in source code.

%second phase on fig 1.1 (tror jeg) - skal være sikker på hvad symbols er
%forklaret mere om den i CSharp > CSharpCodeAnalysis (Portable) > Declarations > Declaration.cs og DeclarationTable.cs (Table indeholder forskellige typer af Declaration, såsom SingleTypeDeclaration)

%second, declration phase: declarations from source and imported metadata are analyzed to form named symbols.
	%the declaration phase as a hierarchical symbol table
	%åbenbart både declaraiton og symbol table?

%***Mangler hvad fase 2 er i fig.1.1, eller hvor det sker præcist.

%*The fourth phase declaration diagnostics (fase 3 i fig 1.1)
%third phase of fig 1.1 (fully bind)
%resultatet er et bound træ, hvilket er en internal repræsentation for Roslyn (som ikke er tilgængelig igennemt API'et)
%symbol visitor sker heri (men den tager et symbol, så det må ske før)

%The fifth IL emitting
%the fourth

%The sixth serialization of portable executable file and pdb file (used for debugging).

%pdb info:
	%http://www.wintellect.com/devcenter/jrobbins/pdb-files-what-every-developer-must-know

%et sted skal scope og type tjek checkes

%secon phase on fig 1.1


%evt. lav en figur over disse faser

%sequential



%\toby[i]{Skal have det et sted, hvor jeg fortæller mere præcist en compiler line billedet om faserne der sker, ligesom beskrevet i blogpostet. Evt. også i forhold til om de køres med en eller flere tråde}
%Earlier in intro we looked at the phases of the compiler in contrast to the compiler API, howerver This does not include all phases. In this section we will explain in more detail what phases the compiler consits of and how it is build on in the code
	%Omformuler så det lyder bedere (men noget i den stil)

%Forklar de forskellige kompilerigns faser

%overordnet beskrivelse af faserne og evt. hvis api i forhold til også
	%ved ikke om jeg skal bruges deres tegning eller lave en ny selv (eller begge dele)

%snak ikke om det under white paper underskriften: API Layers

%MORE:

%Known Limitations and Unimplemented Language Features
	%https://social.msdn.microsoft.com/Forums/vstudio/en-US/f5adeaf0-49d0-42dc-861b-0f6ffd731825/known-limitations-and-unimplemented-language-features?forum=roslyn

%Liste over mulige emner:
	% Beskrivelse af kompileren, og hvad de muliggøre med deres API.
	% De enkelte faser i kompileren og generelt om dens opbygning
	% Hvilke ting de har gjort i forhold til at forbedre performance
	% Fortælle om deres røde og grønne syntaks træ
	% De anvender traditionel compiler teori, med lexer og parser + syntaxtræ og visitor til at traverserer træet. og emit fase.
	%Evt. fortæl om faserne i forhold til mapperne i projektet. Med Core, VB og C# projekternere og til sidst fortæl udfra C#'s mapper (og de forskellige faser).
	%Evt. også nævn syntax visualizer (nok ikke så vigtig, men er et værdifuldt værktøj)
	%Immutable data struktuerer!
	%Hvilke muligheder man har for at implementere STM direkte i Roslyn
		%Ved hjælp af API, hvor vi gør det først og derefter fodrer output til csc.exe
		%Direkte i compileren.


%Den primære brug der er tiltænkt at programmøren skal få adgang er gennem API laget, hvor en del kompleksitet stadig er gemt væk.
	%Vi vil gå direkte ind og ændre i selve compileren, hvilket ikke er tiltænk som hovedtanken fra Microsoft's side.
	
%Evt. skriv at vi vil have fokus på syntax træet og går mest i detalje med det, fordi vores implementering har valgt at ændre i syntax træet.
	%men alligevel kort om de andre faser.
\worksheetend

