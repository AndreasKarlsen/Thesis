\makeatletter \@ifundefined{rootpath}{\input{../../setup/preamble.tex}}\makeatother
\worksheetstart{Preface}{1}{Februar 10, 2015}{Andreas}{../../}
This report documents the master thesis done by group dpt109f15 at the Department of Computer Science at Aalborg University. The thesis was written as part of the Computer Science (IT) study program in the spring of 2015 at the 10th semester.

The first time an acronym is used it will appear in the format: Threads \& Locks (TL). Inline quotations and names will appear in \textit{italics}. The work presented in this report is based on work or results described in books, articles, video lectures and research papers from outside sources. The full list of acronyms along with the bibliography and appendix, can be found at the end of the report.

We would like to give a special thanks to our supervisor Lone Leth Thomsen, from the Department of Computer Science, for excellent guidance and immaculate attention to detail. She supplied invaluable help throughout the project with professionalism and black humour. Her constructive criticism helped us to narrow down the subject and kept us motivated and enthusiastic about the project.

The report is structured with dependencies between the chapters, and the following can be used as a reading guide:
\begin{itemize}
	\item \bsnameref{chap:introduction} presents the motivation for choosing this topic, the related work, the scope, and the hypothesis. Lastly the method of evaluation is presented.
	\item \bsnameref{chap:prelim} establishes the ``locking'' term, and the key concepts of \ac{STM}. This knowledge is required in order to understand the rest of the work.
	\item \bsnameref{chap:roslyn} outlines the structure of the Roslyn compiler, which enables an integration of \ac{STM} into C\#.
	\item \bsnameref{sec:stm_requirements} analyses the requirements to the \ac{STM} system which powers \stmname.
	\item \bsnameref{chap:stm_design} describes the decisions of designing and integrating \stmname based on the requirements.
	\item \bsnameref{chap:implementation} describes the implementation of the \ac{STM} system that powers \stmname, and is based on the requirements and design choices.
	\item \bsnameref{chap:roslyn_extension} describes how Roslyn is extended to encompass language integrated \ac{STM}, thus being a compiler for \stmname.
	\item \bsnameref{chap:evaluation} evaluates \stmname, its associated \ac{STM} library and locking in C\# according to the evaluation method described in \bsref{sec:eval_approach}, which identifies key differences in the concurrency approaches.
	\item Based on this evaluation, a conclusion on the hypothesis is made in \bsnameref{chap:conclusion}.
	 \item To reflect on the decisions made throughout the report, \bsnameref{chap:reflection} discusses the choices made and their consequences.
	 \item Continuation of the work in the future and its potential is discussed in \bsnameref{chap:future_work}.
\end{itemize}\toby{Evt. få de sidste tre punkter til at starte med Chapter X, for consistency}


A general knowledge of C\# and \ac{STM} is assumed. Furthermore, that the reader has knowledge of terminology related to implementation of \ac{STM} system, such as eager vs. lazy updating. If this is not the case a description of the terminology can be found in our previous study\cite[p. 53]{dpt907e14trending}. The reader can skip the description of locking constructs and \ac{STM} key concepts in \bsref{chap:prelim}, if she is familiar with the locking constructs in C\#.\toby{Evt. ryk det her over reading guide - eller bliver det overhovedet nødvendigt? det bliver vel forklaret i reading guide under punkt 2 (der kan man bare tilføje at de kan skippe hvis de ved det)}

\newpage
\vspace*{30 mm}
%\vspace*{\fill}
\begin{vplace}

\begin{minipage}[b]{0.45\textwidth}
 \centering
 \rule{\textwidth}{0.5pt}\\
  Tobias Ugleholdt Hansen\\
 {\footnotesize tuha13@student.aau.dk}
\end{minipage}
\begin{minipage}[b]{0.45\textwidth}
 \centering
 \rule{\textwidth}{0.5pt}\\
  Andreas Pørtner Karlsen\\
 {\footnotesize akarls13@student.aau.dk}
\end{minipage}\\\\
\begin{minipage}[b]{0.45\textwidth}
 \centering
 \rule{\textwidth}{0.5pt}\\
  Kasper Breinholt Laurberg\\
 {\footnotesize klaurb13@student.aau.dk}
\end{minipage}\\\\


\end{vplace}
\worksheetend
