\makeatletter \@ifundefined{rootpath}{\input{../../setup/preamble.tex}}\makeatother
\worksheetstart{Background Knowledge}{0}{Februar 19, 2015}{}{../../}
This chapter contains background knowledge required to understand the following chapters. In \bsref{sec:locking} it accounts for the locking constructs in C\#, establishing the term ``locking''  in respect to our hypothesis. Furthermore, the key concepts of \ac{STM} will be explained in \bsref{chap:stm_key_concepts}, enabling the reader to understand the details, which will be discussed in \bsref{chap:stm_design} and \bsref{chap:implementation}.
\label{chap:prelim}
\section{Locking in C\#}\label{sec:locking}
Locking in C\# can be done with the \bscode{lock} statement, and for special cases the \bscode{Monitor} class can be used which provides additional functionality. Furthermore, a number of other special case locking constructs can be found in the \bscode{System.Threading} namespace e.g. \cite{microsoftSyncPrim} \bscode{Mutex,} \bscode{Semaphore,} \bscode{SpinLock} and \bscode{ReaderWriterLock}. This is not an exhaustive list, but encompasses the constructs used to solve the concurrency problems defined in \bsref{sec:eval_approach} as well as some of the more specialized constructs.
\subsection{Lock Statement}\label{subsec:lock_statement}
The \bscode{lock} statement\cite[p. 102]{csharp2013specificaiton} provides a way to acquire and release a lock on a resource. In the scope following the lock statement, a lock is automatically taken at the beginning and released at the end. The lock provides mutual-exclusion, and other threads trying to acquire a used lock waits until the lock is released. The \bscode{lock} statement ensures that the programmer does not forget to release the lock. \bsref{lst:lock_statement} exemplifies the usage of the lock statement. 
\begin{lstlisting}[label=lst:lock_statement,
  caption={Lock Statement},
  language=Java,  
  showspaces=false,
  showtabs=false,
  breaklines=true,
  showstringspaces=false,
  breakatwhitespace=true,
  commentstyle=\color{greencomments},
  keywordstyle=\color{bluekeywords},
  stringstyle=\color{redstrings},
  morekeywords={atomic, retry, orElse}]  % Start your code-block

  private Object thisLock = new Object();
  public int Get() { 
    lock(thisLock){
      if (!_queue.IsEmpty())
        return _queue.Dequeue();
    }
  }
\end{lstlisting}
\subsection{Monitor}
The \bscode{lock} statement described in \bsref{subsec:lock_statement} is an assisting way of using the \bscode{Monitor} class\cite{msdnMonitor}. The \bscode{Monitor} cannot be instantiated as an object, but can be used in any context through its static methods. The methods \bscode{Enter} and \bscode{Exit} are used to take and release a lock on a resource. The \bscode{Monitor} class also provides additional functionality such as allowing the programmer to specify a timeout of the acquisition of a lock. This is done by using the \bscode{TryEnter} method. The \bscode{Monitor} also facilitates communication between threads with the \bscode{Wait,} \bscode{Pulse,} and \bscode{PulseAll} methods. A thread can call the \bscode{Wait} method to release the lock, thus making it possible for other threads to change the state of the resource. When the other thread is done, it can use \bscode{Pulse} to notify the next waiting thread of changes, and it will then try to reacquire the lock. \bscode{PulseAll} notifies all the threads waiting.
\subsection{Mutex}
The \bscode{Mutex} class\cite{msdnMutex} provides mutual exclusion to a shared resource by allowing only a single thread to acquire the \bscode{Mutex} at a time. The \bscode{Mutex} ensures thread identity,  guaranteeing that only the thread which acquired the \bscode{Mutex} can release it. A \bscode{Mutex} can be acquired by the method \bscode{WaitOne,} which requests ownership of the \bscode{Mutex} and blocks the calling thread until the \bscode{Mutex} is acquired or a supplied timeout is met. When a thread is done using the \bscode{Mutex,} it can use \bscode{ReleaseMutex} to release it.

\subsection{SemaphoreSlim}
The \bscode{SemaphoreSlim} class\cite{msdnSemaphoreSlim} is similar to a \bscode{Mutex,} but can allow multiple threads to enter a shared resource at a time. The amount of concurrent allowed resources is set as a constructor argument. Contrary to \bscode{Mutex,} the \bscode{SemaphoreSlim} does not assure thread identity on the \bscode{WaitOne} or \bscode{Release} methods. Additionally, there is no guaranteed order in which the waiting threads will enter the \bscode{Semaphore}. 

\subsection{SpinLock}
The \bscode{SpinLock} struct\cite{msdnSpinLock} works like a \bscode{Monitor}, and has the methods \bscode{Enter,} \bscode{TryEnter,} and \bscode{Exit} which are called on an instance of the struct. It is implemented as a struct, easing the pressure on the \ac{GC} but requiring the programmer to pass it by ref. The primary purpose is to allow a thread to spin wait, instead of blocking causing a context switch to occur. This is useful in scenarios where locks are fine grained and in large numbers, or when the holding time of the locks is consistently short. It is not reentrant, consequently if the same thread tries to take the lock twice, an exception will be thrown. 

% https://msdn.microsoft.com/en-us/library/System.Threading.SpinLock.aspx
\subsection{ReaderWriterLockSlim}
The class \bscode{ReaderWriterLockSlim}\cite{msdnReaderWriterLockSlim} is a lock with multiple states allowing threads to differentiate between reading or writing to a shared resource. This enables concurrent reads while keeping writes exclusive, and therefore safe. The methods \bscode{EnterReadLock,} \bscode{EnterUpgradeableReadLock,} and \bscode{EnterWriteLock} enter reading, upgrade and write mode respectively. The read mode is only for reading the shared resource, enabling concurrency with other readers, but mutual exclusion to writers. The upgradeable readlock enables a common pattern where a value is read, and only updated under certain conditions. In this case, the lock can be upgraded to a write lock without releasing the read lock, thus disallowing other threads to intervene when changing the lock from read to write. The write lock provides mutual exclusion, ensuring exclusive access to the shared resource. All of these methods are also available in a version which allows a timeout to be specified, and have a corresponding \bscode{Exit} method to release the acquired lock.

\section{STM Key Concepts}
This part contains a modified version of \cite[p. 43-48]{dpt907e14trending}. The reader can skip this section if she is familiar with \ac{STM}.  
\label{chap:stm_key_concepts}

\subsection{Software Transactional Memory}
\ac{STM} provides programmers with a software based transactional model through a library or compiler interface\cite{herlihy2011tm}. It seeks to solve the issues introduced by running multiple processes concurrently in a shared memory space, e.g. race conditions as discussed in \cite[p. 22-26]{dpt907e14trending}. To handle these challenges, \ac{STM} offers the programmer ways to define transaction scopes in critical regions. The transactions are executed concurrently and if successful, changes are committed. If a transaction is not successful it will be retried. Just by defining critical regions, \ac{STM} ensures atomicity and isolation, and as a result the low level details of synchronization are abstracted away from the programmer\cite[p. 48]{harris2005composable}. Therefore, \ac{STM} provides a more declarative approach to handle shared-memory concurrency than by using locks. A number of issues related locking discussed in \cite[p. 26-30]{dpt907e14trending}, such as deadlocks, do not exist in \ac{STM}.

\subsection{Example of using STM}
\label{sec:stm_keyconcepts_example}
Languages supporting \ac{STM} must encompass a language construct for specifying that a section of code should be executed as a transaction and managed by the \ac{STM} system. This basic language construct is often referred to as the \bscode{atomic} block\cite[p. 49]{harris2005composable}\cite[p. 3]{harris2003language}. The \bscode{atomic} block allows programmers to specify a transaction scope wherein code should be executed atomically and isolated, as exemplified in \bsref{lst:stm_atomic_block}. Exactly how a transaction scope is defined varies between \ac{STM} implementations. As an example, an \ac{STM} integrated in the language could look like \bsref{lst:stm_atomic_block} on line 2, while a library based system such as JDASTM\cite{ramadan2009committing} uses calls to the methods \bscode{startTransaction} and \bscode{commitTransaction}.

\begin{lstlisting}[float, label=lst:stm_atomic_block,
  caption={Threadsafe queue},
  language=Java,  
  showspaces=false,
  showtabs=false,
  breaklines=true,
  showstringspaces=false,
  breakatwhitespace=true,
  commentstyle=\color{greencomments},
  keywordstyle=\color{bluekeywords},
  stringstyle=\color{redstrings},
  morekeywords={atomic, retry, orElse}]  % Start your code-block

  public int Get() { 
    atomic {
      if (!_queue.IsEmpty()) {
        return _queue.Dequeue();
      }
    }
  }
\end{lstlisting}

\subsection{Conflicts}
By declaring an atomic block, the programmer gives the responsibility of synchronizing concurrent code to the \ac{STM} system. Avoiding race conditions and deadlocks, while still allowing for optimistic execution introduces conflicts between transactions. In the context of \ac{STM} a conflict are two transactions perform conflicting operations on the same data, resulting in only one of them being able to continue\cite[p. 20]{harris2010transactional}. A conflict arises if one transaction reads the data while the other writes to it. Different techniques of conflict resolution are discussed in \cite[p. 45-46 \& 52-55]{dpt907e14trending}. Despite the different implementation details, the semantics does not change from the view of the programmers. However, it is important to know that transactions may conflict, since a high level of contention can negatively affect performance\cite[p. 52]{dpt907e14trending}.

\subsection{Retry}
By enabling the programmer to interact further with the \ac{STM} system beyond declaring atomic blocks, busy-waiting can be avoided. A common task in concurrent programming is executing code whenever some event occurs. Consider a concurrent queue shared between multiple threads in a producer consumer setup. It is desirable to only have a consumer dequeue an item whenever one is available. Accomplishing this without the need for busy waiting frees the computational resources for other tasks.

In \cite{harris2005composable} Harris et al. introduce the \bscode{retry} statement for assisting in conditional synchronization within the context of \ac{STM}. The \bscode{retry} statement is explicitly placed within an \bscode{atomic} block. If a transaction encounters a retry statement during its execution it indicates that the transaction is not yet ready to run and the transaction should be aborted and retried at some later point\cite[p. 73]{harris2010transactional}. The transaction is not retried immediately but instead blocks, waiting to be awoken when the transaction is retried. The transaction is typically retired when one of the variables read in the transaction is updated by another transaction\cite[p. 51]{harris2005composable}. By blocking the thread instead of repeatedly checking the condition, busy waiting is avoided.

A transaction using the \bscode{retry} statement is shown in \bsref{lst:stm_retry}. If the queue is empty the transaction executes the retry statement of line 4, blocking the transaction until it is retired at a later time.
\begin{lstlisting}[label=lst:stm_retry,
  caption={Queue with retry},
  language=Java,  
  showspaces=false,
  showtabs=false,
  breaklines=true,
  showstringspaces=false,
  breakatwhitespace=true,
  commentstyle=\color{greencomments},
  keywordstyle=\color{bluekeywords},
  stringstyle=\color{redstrings},
  morekeywords={atomic, retry, orElse}]  % Start your code-block
  
  public int Get() {
    atomic {
      if (_queue.IsEmpty()) {
        retry;
      }
      return _queue.Dequeue();
    }
  }
\end{lstlisting}

\subsection{orElse}
In addition to the \bscode{retry} statement Harris et al. propose the \bscode{orElse} block. The \bscode{orElse} block handles the case of waiting on one of many conditions to be true by combining a number of transaction alternatives. These alternatives are evaluated in left-to-right order and only one of the alternatives is committed\cite[p. 52]{harris2005composable}. The \bscode{orElse} block works in conjunction with the \bscode{retry} statement to determine which alternative to execute. An example of a transaction employing the \bscode{orElse} block is shown in \bsref{lst:stm_orelse}. If an alternative executes without encountering a retry statement it gets to commit and the other alternatives are never executed\cite[p. 74]{harris2010transactional}. However, if an alternative encounters a \bscode{retry} statement its memory operations are undone and the next alternative in the chain is executed\cite[p. 74]{harris2010transactional}. If the last alternative encounters a \bscode{retry} statement, the transaction as a whole is blocked awaiting a retry at a later time\cite[p. 74]{harris2010transactional}.

\begin{lstlisting}[label=lst:stm_orelse,
  caption={Queue with orElse},
  language=Java,  
  showspaces=false,
  showtabs=false,
  breaklines=true,
  showstringspaces=false,
  breakatwhitespace=true,
  commentstyle=\color{greencomments},
  keywordstyle=\color{bluekeywords},
  stringstyle=\color{redstrings},
  morekeywords={atomic, retry, orElse}]  % Start your code-block
  
  public int Get() {
    atomic {
      if(_queue.IsEmpty())
        retry;
      return _queue.Dequeue();
    } orElse {
      if(_queue2.IsEmpty())
        retry;
      return _queue2.Dequeue();
    } orElse {
      if(_queue3.IsEmpty())
        retry;
      return _queue3.Dequeue();
    }
  }
\end{lstlisting}

%Transactions
%Strong vs Weak atomicity
%Side-effects
%Syntax
%retry, orElse
%
\worksheetend