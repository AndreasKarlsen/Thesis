\makeatletter \@ifundefined{rootpath}{\input{../../setup/preamble.tex}}\makeatother
\worksheetstart{STM Key Concepts}{0}{Februar 19, 2015}{}{../../}
In the enclosing chapter, we will describe the basic of \ac{STM} as a concept. The purpose is to familiarize the reader with the concepts, and enable her to understand the intriguing details, which will be discussed in \bsref{chap:stm_design} and \andreas{Reference other chapters where we discuss STM details}. The contents of the chapter are a markedly abridged version of \cite[p. 43-48]{dpt907e14trending}. The reader can skip this chapter if he familiar with \ac{STM} or have read \cite[chap. 5]{dpt907e14trending} previously.
\label{chap:stm_key_concepts}

\section{Software Transactional Memory}
\ac{STM} provides programmers with a transactional model, in software, through a library or compiler interface\cite{herlihy2011tm}. It seeks to solve the issues introduced by running multiple processes concurrently in a shared memory space, e.g. race conditions as discussed in \cite[p. 22-26]{dpt907e14trending}. To handle these challenges, \ac{STM} offers the programmer ways to define transaction scopes in critical regions. The transactions are executed concurrently, and if successful, changes are committed. If the transactions are not successful, it will be retried. Just by defining the critical regions, \ac{STM} takes care of ensuring the atomicity and isolation, which means these details are abstracted away from the programmer\cite[p. 48]{harris2005composable}. By ensuring this, \ac{STM} provides a more declarative approach of handling shared-memory concurrency than by using locks. Some issues related to using locks discussed in \cite[p. 26-30]{dpt907e14trending}, such as deadlocks, is non existing in \ac{STM}.

\section{Example of using STM}
\label{sec:stm_keyconcepts_example}
Languages supporting \ac{STM} must encompass some language construct for specifying that a section of code should be executed as a transaction and managed by the \ac{STM} system. This basic language construct is often referred to as the \bscode{atomic} block\cite[p. 49]{harris2005composable}\cite[p. 3]{harris2003language}. The \bscode{atomic} block allows programmers to specify a transaction scope wherein code should be executed atomically and isolated. The atomic block is exemplified in \bsref{lst:stm_atomic_block}. Exactly how a transaction scope is defined, varies between \ac{STM} implementations. As an example, an \ac{STM} integrated in the language could look like \bsref{lst:stm_atomic_block} on line 2, while a library based system such as JDASTM\cite{ramadan2009committing} uses calls to the methods \bscode{startTransaction} and \bscode{commitTransaction}.

\begin{lstlisting}[label=lst:stm_atomic_block,
  caption={Threadsafe queue},
  language=Java,  
  showspaces=false,
  showtabs=false,
  breaklines=true,
  showstringspaces=false,
  breakatwhitespace=true,
  commentstyle=\color{greencomments},
  keywordstyle=\color{bluekeywords},
  stringstyle=\color{redstrings},
  morekeywords={atomic, retry, orElse}]  % Start your code-block

  public int Get() { 
    atomic {
      if (!_queue.IsEmpty()) {
        return _queue.Dequeue();
      }
    }
  }
\end{lstlisting}

\section{Conflicts}
By declaring an atomic block, the programmer gives the responsibility of synchronizing concurrent code correctly to the \ac{STM} system. Avoiding race conditions and deadlocks, while still allowing for optimistic execution introduces conflicts between transactions. In the context of \ac{STM} a conflict is two transactions performing conflicting operations on the same data, resulting in only one of them being able to continue\cite[p. 20]{harris2010transactional}. A conflict arises if two transactions attempt to write to the same data, or if one transaction reads the data while the other writes to it. Different techniques of conflict resolution are discussed in \cite[p. 45-46 \& 52-55]{dpt907e14trending}. Despite the different implementation details, it does not change the semantics from the programmers point of view. However, it is important to know that transactions can conflict, since it will cause contention and thus negatively affect performance.

\section{Retry}
By enabling the programmer to interact further with the \ac{STM} system than declaring atomic blocks, busy-waiting can be avoided. A common task in concurrent programming is executing code whenever some event occurs. Consider a concurrent queue shared between multiple threads in a producer consumer setup. It will be useful to only have a consumer dequeue an item whenever one is available. Accomplishing this without the need for busy waiting would also be preferable.

In \cite{harris2005composable} Harris et al. introduce the \bscode{retry} statement for assisting in conditional synchronization within the context of \ac{STM}. The \bscode{retry} statement is explicitly placed by programmers within an \bscode{atomic} block. If a transaction encounters a retry statement during its execution it indicates that the transaction is not yet ready to run and the transaction should be aborted and retried at some later point\cite[p. 73]{harris2010transactional}. The transaction is not retried immediately but instead blocks, waiting to be awoken when the transaction is to be retried. The transaction is typically retired when one of the variables read in the transaction is updated by another transaction\cite[p. 51]{harris2005composable}. By blocking the thread instead of repeatedly checking the condition, busy waiting is avoided.

A transaction using the \bscode{retry} statement is shown in \bsref{lst:stm_retry}. If the queue is empty the transaction executes the retry statement of line 3, blocking the transaction until it is retired at a later time.
\begin{lstlisting}[label=lst:stm_retry,
  caption={Queue with retry},
  language=Java,  
  showspaces=false,
  showtabs=false,
  breaklines=true,
  showstringspaces=false,
  breakatwhitespace=true,
  commentstyle=\color{greencomments},
  keywordstyle=\color{bluekeywords},
  stringstyle=\color{redstrings},
  morekeywords={atomic, retry, orElse}]  % Start your code-block
  
  public int Get() {
    atomic {
      if (_queue.IsEmpty()) {
        retry;
      }
      return _queue.Dequeue();
    }
  }
\end{lstlisting}

\section{orElse}
In addition to the \bscode{retry} statement Harris et al. propose the \bscode{orElse} block. The \bscode{orElse} block handles the case of waiting on one of many conditions to be true by combining a number of transaction alternatives. The alternatives are evaluated in left-to-right order and only one of the alternatives is committed\cite[p. 52]{harris2005composable}. The \bscode{orElse} block works in conjunction with the \bscode{retry} statement to determine which alternative to execute. An example of a transaction employing the \bscode{orElse} block is shown in \bsref{lst:stm_orelse}. If an alternative executes without encountering a retry statement it gets to commit and the other alternatives are never executed\cite[p. 74]{harris2010transactional}. If an alternative however encounters a \bscode{retry} statement its memory operations are undone and the next alternative in the chain is executed\cite[p. 74]{harris2010transactional}. If the last alternative encounters a \bscode{retry} statement the transaction as a whole is blocked awaiting a retry at a later time\cite[p. 74]{harris2010transactional}.

\begin{lstlisting}[label=lst:stm_orelse,
  caption={Queue with orElse},
  language=Java,  
  showspaces=false,
  showtabs=false,
  breaklines=true,
  showstringspaces=false,
  breakatwhitespace=true,
  commentstyle=\color{greencomments},
  keywordstyle=\color{bluekeywords},
  stringstyle=\color{redstrings},
  morekeywords={atomic, retry, orElse}]  % Start your code-block
  
  public int Get() {
    atomic {
      if(_queue.IsEmpty())
        retry;
      return _queue.Dequeue();
    } orElse {
      if(_queue2.IsEmpty())
        retry;
      return _queue2.Dequeue();
    } orElse {
      if(_queue3.IsEmpty())
        retry;
      return _queue3.Dequeue();
    }
  }
\end{lstlisting}

%Transactions
%Strong vs Weak atomicity
%Side-effects
%Syntax
%retry, orElse
%
\worksheetend